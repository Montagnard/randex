\documentclass[12pt, leqno]{exam}
\qformat{\textbf{Exercice \thequestion}:\hfill}
\usepackage[x11names]{xcolor}
\usepackage{pstricks}
\usepackage{amsfonts,amsmath,amssymb,mathabx,mathtools, mathrsfs,enumitem,bm,slashed}
\usepackage{geometry}
\usepackage{tikz-cd}
\usetikzlibrary[arrows]
\usepackage[T1]{fontenc}
\usepackage{lmodern}
\usepackage[utf8]{inputenc}
\usepackage[french]{babel}
\usepackage[autostyle]{csquotes}
\usepackage{multicol}
\usepackage[]{hyperref}
\hypersetup{
    colorlinks=false,
}


\header{Algèbre linéaire et bilinéaire}{TD }{6 avril 2018}

\begin{document}
\def\H{\mathbb{H}}
\def\P{\mathbf{P}}
\def\Q{\mathbb{Q}}
\def\R{\mathbb{R}}
\def\id{I}
\newcommand{\norm}[1]{  \Vert #1 \Vert }

\begin{questions}
\question
\begin{parts}
  \part Pour chacune des matrices suivantes, déterminer ses valeurs propres
  \part Donner une base de chaque espace propre
  \part La matrice est-elle diagonalisable ? Si oui, écrire la matrice sous la forme $PDP^{-1}$ avec $D$ matrice diagonale. Si non, sous quelle forme peut-on quand même mettre la matrice ?
\end{parts}
\begingroup
\allowdisplaybreaks
\begin{align*}
  \#{This is a long-form Jinja comment}
  &
\BLOCK{ for x in range(1,n_exo) }
\text{\hyperlink{mat_\VAR{x}}{\VAR{x})}}
  \VAR{latex_matrix(list_matrix[x].matrix)}
  \BLOCK{ if x is divisibleby(3) } \\ &\BLOCK{ else } && \BLOCK{ endif }  \BLOCK{ endfor }
\end{align*}
\endgroup
\end{questions}

\BLOCK{ for x in range(1,n_exo) }
\par
\hypertarget{mat_\VAR{x}}{\VAR{x})}
Le polynome caracteristique de la matrice est donne par
$$ \VAR{list_matrix[x].latex_poly("x")} = \VAR{list_matrix[x].poly_factored_latex("x")}$$

\noindent On calcule la dimensions des espaces propres :

\BLOCK{ for root,mult in list_matrix[x].mult_geo.items() }

$$ M \BLOCK{ if root < 0 } + \VAR{0-root} \BLOCK{ else } - \VAR{root} \BLOCK{ endif }
\cdot Id = \VAR{latex_matrix(list_matrix[x].shifted_matrix(root))} $$
Et on a
$dim(E_{\VAR{root}}) =  \VAR{mult} $.

Une base est donnée par la famille 
$\BLOCK{ for vect in list_matrix[x].basis_eigenspace(root) }
\begin{pmatrix} \VAR{vect} \end{pmatrix}
\BLOCK{ endfor }$

\BLOCK{ endfor }

On est donc dans le cas où M
\BLOCK{ if list_matrix[x].is_diago } est
\BLOCK{ else }  n'est pas
\BLOCK{ endif } diagonalisable.
On peut ecrire par exemple la matrice sous la forme :

$$
\VAR{ latex_matrix(list_matrix[x].passage) } \cdot
\VAR{ latex_matrix(list_matrix[x].jordan) } \cdot
\VAR{ latex_matrix(list_matrix[x].inverse) } $$


\BLOCK{ endfor }

\end{document}

