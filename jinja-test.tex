\documentclass[12pt, leqno]{exam}
\qformat{\textbf{Exercice \thequestion}:\hfill}
\usepackage[x11names]{xcolor}
\usepackage{pstricks}
\usepackage{amsfonts,amsmath,amssymb,mathabx,mathtools, mathrsfs,enumitem,bm,slashed}
\usepackage{geometry}
\usepackage{tikz-cd}
\usetikzlibrary[arrows]
\usepackage[T1]{fontenc}
\usepackage{lmodern}
\usepackage[utf8]{inputenc}
\usepackage[french]{babel}
\usepackage[autostyle]{csquotes}
\usepackage{multicol}
\usepackage[]{hyperref}
\hypersetup{
    colorlinks=false,
}


\header{Algèbre linéaire et bilinéaire}{TD }{6 avril 2018}

\begin{document}
\def\H{\mathbb{H}}
\def\P{\mathbf{P}}
\def\Q{\mathbb{Q}}
\def\R{\mathbb{R}}
\def\id{I}
\newcommand{\norm}[1]{  \Vert #1 \Vert }

\begin{questions}
\question
\begin{parts}
  \part Pour chacune des matrices suivantes, déterminer ses valeurs propres
  \part Donner une base de chaque espace propre
  \part La matrice est-elle diagonalisable ? Si oui, écrire la matrice sous la forme $PDP^{-1}$ avec $D$ matrice diagonale. Si non, sous quelle forme peut-on quand même mettre la matrice ?
\end{parts}
\begingroup
\allowdisplaybreaks
\begin{align*}
  \#{This is a long-form Jinja comment}
  &
\BLOCK{ for x in range(1,100) }
\text{\hyperlink{mat_\VAR{x}}{\VAR{x})}}
  \VAR{list_matrix[x]}
  \BLOCK{ if x is divisibleby(3) } \\ &\BLOCK{ else } && \BLOCK{ endif }  \BLOCK{ endfor }
\end{align*}
\endgroup
\end{questions}

\BLOCK{ for x in range(1,100) }
\par
\hypertarget{mat_\VAR{x}}{\VAR{x})}
\VAR{list_sol[x]}  
\BLOCK{ endfor }

\end{document}

